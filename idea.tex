% Created 2024-09-24 Tue 16:13
% Intended LaTeX compiler: xelatex
\documentclass[11pt]{article}
\usepackage{graphicx}
\usepackage{longtable}
\usepackage{wrapfig}
\usepackage{rotating}
\usepackage[normalem]{ulem}
\usepackage{capt-of}
\usepackage{hyperref}
\usepackage[cache=false]{minted}
\author{Samuel Grahn}
\date{\today}
\title{Idea}
\hypersetup{
 pdfauthor={Samuel Grahn},
 pdftitle={Idea},
 pdfkeywords={},
 pdfsubject={},
 pdfcreator={Emacs 29.4 (Org mode 9.7.11)}, 
 pdflang={English}}
\usepackage{biblatex}

\begin{document}

\maketitle
\tableofcontents

A state consists of two integers, \(must\) and \(may\).
\(must\) signifies the number of values that \textbf{must} be present at this event, and
\(may\) signifies the number of values that \textbf{may} be present.

A final state of monitoring consists of one state per event. Namely, \(s_{!c}\), \(s_{!r}\), \(s_{?c}\), \(s_{{?r}}\)

Denote the minimum integer appearing in \(s_{{!c}}\) and \(s_{!r}\) by \(min_!\). similarly for \(max\), and for \(?\).
e.g.

a: [0,1][1,1][0,1][0,0]
says that when push(a) is called, there are between 0 and 1 elements in the stack, and when it returns there is exactly 1 element in the stack. Similarly for pop.

We say a value is compatible with a number \(n\) if
\(n \in [min_!^a, max_!^a] \cap [ min_?^a, max_?^a]\).

A chain \(c_n\) is a sequence of integers, with \(c_0 = 0\), such that each pair of consecutive numbers has a difference of at most 1.

A history is linearizable iff there is sequence \((c_0, v_0), \dots, (c_n, v_n)\), equal in size to the history, such that
\(c_0, \dots, c_n\) form a chain, and each \(v_i\) is compatible with \(c_i\).
\end{document}
